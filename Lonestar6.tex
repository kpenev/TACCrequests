\documentclass{article}

\usepackage[bookmarks=true]{hyperref}

\title{Measuring Tidal Dissipation of Low Mass Stars and Giant Planets}

\begin{document}

\maketitle

\section{Project Description}

Tides play an important role in shaping exoplanet and binary star systems. The
effects of tidal coupling are evident over the entire life cycle of binary stars
and exoplanet systems, and have been suggested to play a key role in the
formation of short period giant planets, ultra short period planets, and a wide
range of satellite systems within the Solar System. In spite of the wide spread
implications, there is limited understanding of the processes involved, and the
strength of the tidal coupling they produce. The proposed investigation aims to
provide systematic, consistent, and reliable measurements of tidal coupling over
a broad range of parameters, directly usable to model the effect of tides for
almost all known exoplanet systems.

Thanks current and past allocations on stampede 2 we have been able to complete
all calculations for four publications in refereed journals, present results at
four conferences, and generate a Ph.D. dissertation (successfully defended in
Fall 2022). The current allocation on Lonestar6 has allowed us to complete all
analysis for one more publication, the article text is currently being
finalized, and we expect to submit within a month. Additional work is on-going
on two more publication, one of which we expect to complete within the next 3
months (using Lonestar6 resources we are requesting in this proposal) and the
other within 6 to 9 months.

\section{Resource Justification/Computational Plan}

We plan to perform Bayesian analysis to constrain a generic parametrized tidal
dissipation model against observations of the eccentricity and spin
distributions of binary star and exoplanet systems. In particular we will use
Markov Chain Monte Carlo, using the \texttt{emcee}
package\footnote{\url{https://emcee.readthedocs.io/en/stable/}}.

We have fully developed and are already using all the code for the analysis that
we are requesting additional allocation for on both Lonestar6 and Stampede2. Our
applications require only small amount of RAM and rely on widely used and
extensively tested/benchmarked open source packages for parallelizing
computations.

We have begun the eccentricity analysis of 75 exoplanet systems in order to
measure the tidal friction in gas giant planets. Using 1\,000 SUs of the current
Lonestar6 allocation, we were able to generate roughly 4\% of the final number
of Markov chain samples required to reliably probe the probability distribution
of model parameters, hence we expect the complete analysis will require
approximately 25\,000 SUs. The project currently has 5\,000 SUs remaining of the
current allocation, hence we are requesting an additional 20\,000 SUs to
complete the project.

We described this particular application as one of the future projects we plan
to pursue using LS6 in the original allocation request predicting it would
require 24\,000 SUs (almost identical to the updated estimate above).

It This is almost exactly the number of SUs we predicted this project will take
in the original proposal (24\,000SUs vs 25\,000SUs). Similarly, the already
completed project with the current allocation also matched almost exactly the
allocation requested.

\end{document}
