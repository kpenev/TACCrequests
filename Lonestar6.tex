\documentclass{article}

\renewcommand\rmdefault{phv}

\usepackage{import}
\usepackage{comment}
\usepackage{soul}
%%GB: General commands to make editing easier and with less typos.

% -----------------------------------------------
\newcommand{\tbd}[1]{{\par\bf\textsc{TBD: #1\\}}}
\newcommand{\ctbd}[1]{}
\newcommand{\cor}{\textcolor{red}{(corr?) }}
\newcommand{\spl}{\textcolor{red}{(spl?) }}

\newcommand{\ii}{\'\i }
\newcommand{\oo}{\H{o}}
\newcommand{\uu}{\H u}

% --------------------------------------
\newcommand{\lc}{light curve}
\newcommand{\lcs}{light curves}
\newcommand{\Lc}{Light curve}
\newcommand{\Lcs}{Light curves}
\newcommand{\avg}[1]{\ensuremath{\langle #1\rangle}}
\newcommand{\dpt}{data-point}
\newcommand{\dpts}{data-points}
\newcommand{\tel}{telescope}
\newcommand{\magn}{magnitude}
\newcommand{\stan}{standard}
\newcommand{\aper}{aperture}
\newcommand{\oot}{out-of-transit}
\newcommand{\OOT}{Out-of-Transit}
\newcommand{\cfa}{Harvard-Smithsonian Center for Astrophysics (CfA)}
\newcommand{\cfadigi}{CfA Speedometers}
\newcommand{\cmd}{color-magnitude diagram}

% ---------------------------------------------
\newcommand{\C}{\ensuremath{^{\circ}C\;}}
\newcommand{\el}{\ensuremath{e^-}}
\newcommand{\sqarcsec}{\ensuremath{\Box^{\prime\prime}}}
\newcommand{\sqarcdeg}{\ensuremath{\Box^{\circ}}}
\newcommand{\pxs}{\ensuremath{\rm \arcsec pixel^{-1}}}
\newcommand{\conc}[1]{\noindent\par{\noindent{$\mathbf \Longrightarrow$ \bf #1}}}
\newcommand{\aduel}{\ensuremath{\lbrack ADU/\el \rbrack}}
\newcommand{\eladu}{\ensuremath{\lbrack \el/ADU \rbrack}}
\newcommand{\adupixs}{\ensuremath{\rm ADU/(pix\, s)}}
\newcommand{\diam}{\ensuremath{\oslash}}
\newcommand{\ccdsize}[1]{\ensuremath{\rm #1\times\rm#1}}
\newcommand{\tsize}[1]{\mbox{\rm #1 m}}
\newcommand{\ghr}{\ensuremath{^h}}
\newcommand{\gmin}{\ensuremath{^m}}
\newcommand{\Ks}{\ensuremath{K_s}}
\newcommand{\masy}{\ensuremath{\rm mas\,yr^{-1}}}
\newcommand{\kms}{\ensuremath{\rm km\,s^{-1}}}
\newcommand{\ms}{\ensuremath{\rm m\,s^{-1}}}

\newcommand{\hd}[1]{\mbox{HD #1}}

% ---------------------------------------------------------------------
\newcommand{\teff}{\ensuremath{T_{\rm eff}}}
\newcommand{\logg}{\ensuremath{\log{g}}}
\newcommand{\vsini}{\ensuremath{v \sin{i}}}
\newcommand{\feh}{[Fe/H]}
\newcommand{\logl}{\ensuremath{\log{L}}}

\newcommand{\rsun}{\ensuremath{R_\sun}}
\newcommand{\msun}{\ensuremath{M_\sun}}
\newcommand{\lsun}{\ensuremath{L_\sun}}
\newcommand{\teffsun}{\ensuremath{T_{eff,\sun}}}
\newcommand{\rhosun}{\ensuremath{\rho_\sun}}

\newcommand{\rstar}{\ensuremath{R_*}}
\newcommand{\mstar}{\ensuremath{M_*}}
\newcommand{\lstar}{\ensuremath{L_*}}
\newcommand{\loglstar}{\ensuremath{\log{L_*}}}
\newcommand{\teffstar}{\ensuremath{T_{eff,*}}}
\newcommand{\rhostar}{\ensuremath{\rho_*}}

\newcommand{\rearth}{\ensuremath{R_\earth}}
\newcommand{\mearth}{\ensuremath{M_\earth}}
\newcommand{\learth}{\ensuremath{L_\earth}}
\newcommand{\teffearth}{\ensuremath{T_{eff,\earth}}}
\newcommand{\rhoearth}{\ensuremath{\rho_\earth}}

%\newcommand{\rpl}{\ensuremath{R_{\rm P}}}
%\newcommand{\mpl}{\ensuremath{M_{\rm P}}}
%\newcommand{\lpl}{\ensuremath{L_{\rm P}}}
%\newcommand{\teffpl}{\ensuremath{T_{eff,{\rm P}}}}
%\newcommand{\rhopl}{\ensuremath{\rho_{\rm P}}}
%\newcommand{\ipl}{\ensuremath{i_{\rm P}}}

\newcommand{\rpl}{\ensuremath{R_{P}}}
\newcommand{\mpl}{\ensuremath{M_{P}}}
\newcommand{\lpl}{\ensuremath{L_{P}}}
\newcommand{\teffpl}{\ensuremath{T_{eff,{P}}}}
\newcommand{\rhopl}{\ensuremath{\rho_{P}}}
\newcommand{\ipl}{\ensuremath{i_{P}}}
\newcommand{\gpl}{\ensuremath{g_{P}}}

\newcommand{\rjup}{\ensuremath{R_{\rm J}}}
\newcommand{\mjup}{\ensuremath{M_{\rm J}}}
\newcommand{\ljup}{\ensuremath{L_{\rm J}}}
\newcommand{\teffjup}{\ensuremath{T_{eff,{\rm J}}}}
\newcommand{\rhojup}{\ensuremath{\rho_{\rm J}}}
\newcommand{\gjup}{\ensuremath{\g_{\rm J}}}

\newcommand{\msini}{\ensuremath{m \sin i}}

% ---------------------
\newcommand{\ordo}{\ensuremath{\mathcal{O}}}

% -----------------------------
\newcommand{\pack}[1]{\textsc{\lowercase{#1}}}
\newcommand{\prog}[1]{\texttt{\lowercase{#1}}}
\newcommand{\iraf}{\pack{iraf}}
\newcommand{\todcor}{\prog{todcor}}
\newcommand{\xcsao}{\prog{xcsao}}
\newcommand{\daophot}{\pack{daophot}}
\newcommand{\fihat}{\pack{fihat}}
\newcommand{\fistar}{\prog{fistar}}
\newcommand{\fiphot}{\prog{fiphot}}
\newcommand{\grmatch}{\prog{grmatch}}
\newcommand{\grtrans}{\prog{grtrans}}

% ---------------------------------------
\newcommand{\pref}[1]{p.~\pageref{#1}}
\newcommand{\figr}[1]{Fig.~\ref{fig:#1}}
\newcommand{\secr}[1]{\mbox{\S\ \ref{sec:#1}}}
\newcommand{\eqr}[1]{Eq.~\ref{eq:#1}}
\newcommand{\tabsr}[1]{Tab.~\ref{tab:#1}}
\newcommand{\tabr}[1]{\mbox{Table~\ref{tab:#1}}}
\newcommand{\figrp}[1]{Fig.~\ref{fig:#1} on \pref{fig:#1}}
\newcommand{\secrp}[1]{\S\ref{sec:#1} on \pref{sec:#1}}
\newcommand{\eqrp}[1]{Eq.~\ref{eq:#1} on \pref{eq:#1}}
\newcommand{\tabrp}[1]{Tab.~\ref{tab:#1} on \pref{tab:#1}}

% --------------------------------------
%
% Instruments
%
% FLWO 1.2 m telescope
\newcommand{\flwof}{\mbox{FLWO 1.2 m}}

% FLWO 1.5 m telescope
\newcommand{\flwos}{\mbox{FLWO 1.5 m}}

% TopHAT 0.25m telescope
\newcommand{\flwot}{\mbox{TopHAT 0.25 m}}

% MMT
\newcommand{\mmt}{\mbox{MMT 6.5 m}}

% Spitzer
\newcommand{\ssts}{{\em Spitzer}}
\newcommand{\sstL}{{\em Spitzer Space Telescope}}

% HST
\newcommand{\hst}{{\em HST}}


% --------------------------------------
% Variable types
%
\newcommand{\dscu}{\mbox{$\delta$ Scuti}}
\newcommand{\gdor}{\mbox{$\gamma$ Dor}}

\newcommand\kepler{\textit{Kepler}}
\newcommand\tess{TESS}
\newcommand{\ktwo}{\emph{K2}}
\newcommand{\plato}{PLATO}

\newcommand{\refsec}[1]{\mbox{\S\ \ref{sec:#1}}}
\newcommand{\fig}[1]{Fig.\,\ref{fig:#1}}

% --------------------------------------
% Formatting
%
\newcommand{\fancysection}[1]{
%
    \section{#1}
%
    \hrule \vspace{0.5mm}
%
}

\newcommand{\hlinesep}{\noindent\rule{\linewidth}{1pt}\\\vspace{-5mm}}

\newcommand{\subpixtools}{\pack{SuperPhot}}
\newcommand\fitsubpix{\pack{FitSubPix}}
\newcommand\fitpsf{\pack{FitPSF}}
\newcommand\subpixphot{\pack{SubPixPhot}}
\newcommand\imsubtract{\pack{ImSubtract}}

\newcommand\hasfeature{\cellcolor{green}\checkmark}
\newcommand\nofeature{\cellcolor{red!60}$\mathbf{\times}$}
\newcommand\mostlyfeature{\cellcolor{green!20!yellow}mostly}
\newcommand\somefeature{\cellcolor{orange}some}
\newcommand\partialfeature[1]{\color{red}{#1}}

\newcommand\poet{\texttt{POET}}

\newcommand\secinputdata{\refsec{input_data}}
\newcommand\sectestingtheory{\refsec{testing_theory}}
\newcommand\secmethodology{\refsec{methodology}}
\newcommand\secprelimresults{\refsec{preliminary_results}}
\newcommand\ruskin{\citep{Patel_Penev_21}}
\newcommand\ebmcmc{\citep{Windemuth_Agol_Kiefer_19}}


\usepackage[font={normal,it}]{caption}
\renewcommand{\familydefault}{\sfdefault}

\usepackage{floatrow}
\usepackage{wrapfig}
\usepackage[pdftex]{graphics,graphicx}
\usepackage{epstopdf}
\graphicspath{ {project_description/figures/} }
\usepackage{pslatex}
\usepackage[margin=2.5cm]{geometry}
\usepackage{fancyhdr}
\usepackage{sectsty}
\allsectionsfont{\normalsize}
\usepackage{natbib,amsmath,amssymb}
\usepackage{sidecap}
\usepackage{aas_macros}
\usepackage{multicol}
\usepackage{relsize}
\usepackage[bookmarks=true]{hyperref}
\usepackage{ifthen}
\usepackage[table]{xcolor}
\usepackage{rotating}
\usepackage{booktabs}

\usepackage{tabularx}
\usepackage{makecell}

\usepackage{sidecap}

\setcounter{secnumdepth}{3}
\setcounter{tocdepth}{2}
\usepackage{multirow}
\usepackage{spverbatim}
\usepackage{pdfpages}

\usepackage{enumitem}
\setlist{nosep}
\setlist[itemize]{leftmargin=*}
\setlist[enumerate]{leftmargin=*}

\usepackage[compact]{titlesec}
\titlespacing*{\section}{0pt}{0pt}{0pt}
\titlespacing*{\subsection}{0pt}{0pt}{0pt}
\titlespacing*{\subsubsection}{0pt}{0pt}{0pt}
%\titlelabel{\roman{\thetitle}.\quad}
%\titleformat{\paragraph}
%    {\normalfont\bfseries}
%    {\theparagraph\quad}
%    {0pt}
%	{\vspace{-3mm}}

\widowpenalty 0
\widowpenalties 0
\clubpenalties 1 0

\renewcommand\bibsection{}

\chead{}

\rhead{\thepage}

\lfoot{}

\cfoot{}

\rfoot{}

\newlength{\defaultparfillskip}
\setlength{\defaultparfillskip}{\parfillskip}

\newlength{\defaultparindent}
\setlength{\defaultparindent}{\parindent}

\definecolor{highlightcolor}{rgb}{1.00, 0.50, 0.50}%{0.0274, 0.7215, 0.5019}

\sethlcolor{highlightcolor}

%\renewcommand{\hl}[1]{\pdfliteral direct {2 Tr 0.5 w}#1\pdfliteral direct {0 Tr 0 w}}

\title{\vspace{-20mm}Measuring Tidal Dissipation of Low Mass Stars and Giant Planets}
\date{}


\begin{document}

\maketitle

\vspace{-15mm}

\section{Project Description}

Tides play an important role in shaping exoplanet and binary star systems. The
effects of tidal coupling are evident over the entire life cycle of binary stars
and exoplanet systems, and have been suggested to play a key role in the
formation of short period giant planets, ultra short period planets, and a wide
range of satellite systems within the Solar System. In spite of the wide spread
implications, there is limited understanding of the processes involved, and the
strength of the tidal coupling they produce. The proposed investigation aims to
provide systematic, consistent, and reliable measurements of tidal coupling over
a broad range of parameters, directly usable to model the effect of tides for
almost all known exoplanet systems.

The Lonestar6 allocation for the past year has allowed us to complete all
analysis for two more publication. One article was submitted to the Monthly
Notices of the Royal Astronomical Society (MNRAS, one of the premier
astrophysics journals) on April 4 2023 and we are hoping a peer review report
will be provided soon. The second publication has a complete write-up and is
undergoing final round of reviews from all authors prior to submission (also to
MNRAS). We expect submitting within the next 2 weeks. Both articles acknowledge
use of TACC resources.

We request resources for two projects that we expect to complete over the
following year:

\textbf{Project 1:} We are currently in advanced stages of the development of
the software for a project that will study tides in a collection of 184 binary
stars that orbit each other close enough to raise significant tides on each
other. The friction these tides experience change the orbits of the binaries
over time and we plan to use these signatures to infer the amount of tidal
friction present.  We will use Bayesian analysis (Markov chain Monte Carlo or
MCMC) to constrain a generic parametric model of tidal friction using
observations of the eccentricity of the orbits of these binary stars.

\textbf{Project 2:} We are also in the beginning stages of the software
development for a project that aims to prepare an approximately 10 times larger,
but otherwise very similar catalog to the one we above. The catalog will be
based on observations from space (from the TESS satellite) of the dimming of the
stars as they block each other's light. Thanks to the unprecedented precision of
the observations and the fact that TESS has observed the entire sky, this
catalog will provide a unique combination of very precisely determined physical
parameters for a very large collection of very bright stars. We plan to use this
catalog to explore even more tidal physics (though not during the following
year), but it will also provide invaluable input data for studies by many other
research groups as is evidenced by the much smaller catalog mentioned above.
This project will also rely on MCMC analysis, this time using models of the
eclipses and other effects detectable in these binary stars to compare to the
observed combined brightness of each system by TESS.

\section{Resource Justification/Computational Plan}

We will use the \texttt{emcee}
package\footnote{\url{https://emcee.readthedocs.io/en/stable/}} for the MCMC
analysis. Both applications described above require only small amount of RAM and
relies on widely used and extensively tested/benchmarked open source packages
for parallelizing the computations.

We have developed and are already using much of the code for project 1,
including for the two publications produced with last year's allocation.  The
remaining modifications will not affect the computational requirements. In fact
the computing resources per binary star will be almost identical to those of the
recently submitted article, which consumed 20\,000 SUs on Lonestar6 for the
analysis of 70 binaries. We already have all the data for the proposed tidal
analysis at hand -- 184 binaries -- simple scaling shows 53\,000 SUs will be
required for the successful completion of project 1.

The requirements of project 2 are far less straightforward to estimate since so
far we only implemented parts of the calculations required. Nonetheless we that
allows us to obtain a lower limit. The computationally expensive part of the
modeling consists of two parts: modeling the eclipses of the two stars and
modeling their out of eclipse variability. We have a model for the former, which
we estimate should be the more computationally expensive task. A similar
analysis was performed by \citet{Windemuth_et_al_19} using data from a different
satellite. They showed $10^6$ \texttt{emcee} steps of 128 coupled MCMC chains
were required for the samples to converge to a stable posterior distribution. We
carried out several two hour MCMC runs on the Lonestar6 debugging partition
finding that approximately 13\,000 MCMC steps per hour are produced using 8
parallel processes for a single binary star. We will run 16 stars per node each
using 8 parallel processes. We found that using up to 8 processes resulted in 7
times more steps per hour than using only 1 process. Hence, we were making
efficient use of the node. With over 1500 systems to analyze, we estimate
at least 9\,000 SUs will be required for models including only eclipses. We
request 12\,000 SUs approximately accounting for the less computationally
expensive but non-negligible modeling of out-of-eclipse variability.

Thus we request a combined 65\,000 SUs to support both projects described above.

\section*{\centering{\Large References and Citations}}
\addcontentsline{toc}{section}{\protect{\numberline{\thesection}}References and
Citations}

\bibliography{bibliography}

\bibliographystyle{mnras}

\end{document}
