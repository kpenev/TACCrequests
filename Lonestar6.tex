\documentclass[12pt]{article}

\renewcommand\rmdefault{phv}

\usepackage{import}
\usepackage{comment}
\usepackage{soul}
%%GB: General commands to make editing easier and with less typos.

% -----------------------------------------------
\newcommand{\tbd}[1]{{\par\bf\textsc{TBD: #1\\}}}
\newcommand{\ctbd}[1]{}
\newcommand{\cor}{\textcolor{red}{(corr?) }}
\newcommand{\spl}{\textcolor{red}{(spl?) }}

\newcommand{\ii}{\'\i }
\newcommand{\oo}{\H{o}}
\newcommand{\uu}{\H u}

% --------------------------------------
\newcommand{\lc}{light curve}
\newcommand{\lcs}{light curves}
\newcommand{\Lc}{Light curve}
\newcommand{\Lcs}{Light curves}
\newcommand{\avg}[1]{\ensuremath{\langle #1\rangle}}
\newcommand{\dpt}{data-point}
\newcommand{\dpts}{data-points}
\newcommand{\tel}{telescope}
\newcommand{\magn}{magnitude}
\newcommand{\stan}{standard}
\newcommand{\aper}{aperture}
\newcommand{\oot}{out-of-transit}
\newcommand{\OOT}{Out-of-Transit}
\newcommand{\cfa}{Harvard-Smithsonian Center for Astrophysics (CfA)}
\newcommand{\cfadigi}{CfA Speedometers}
\newcommand{\cmd}{color-magnitude diagram}

% ---------------------------------------------
\newcommand{\C}{\ensuremath{^{\circ}C\;}}
\newcommand{\el}{\ensuremath{e^-}}
\newcommand{\sqarcsec}{\ensuremath{\Box^{\prime\prime}}}
\newcommand{\sqarcdeg}{\ensuremath{\Box^{\circ}}}
\newcommand{\pxs}{\ensuremath{\rm \arcsec pixel^{-1}}}
\newcommand{\conc}[1]{\noindent\par{\noindent{$\mathbf \Longrightarrow$ \bf #1}}}
\newcommand{\aduel}{\ensuremath{\lbrack ADU/\el \rbrack}}
\newcommand{\eladu}{\ensuremath{\lbrack \el/ADU \rbrack}}
\newcommand{\adupixs}{\ensuremath{\rm ADU/(pix\, s)}}
\newcommand{\diam}{\ensuremath{\oslash}}
\newcommand{\ccdsize}[1]{\ensuremath{\rm #1\times\rm#1}}
\newcommand{\tsize}[1]{\mbox{\rm #1 m}}
\newcommand{\ghr}{\ensuremath{^h}}
\newcommand{\gmin}{\ensuremath{^m}}
\newcommand{\Ks}{\ensuremath{K_s}}
\newcommand{\masy}{\ensuremath{\rm mas\,yr^{-1}}}
\newcommand{\kms}{\ensuremath{\rm km\,s^{-1}}}
\newcommand{\ms}{\ensuremath{\rm m\,s^{-1}}}

\newcommand{\hd}[1]{\mbox{HD #1}}

% ---------------------------------------------------------------------
\newcommand{\teff}{\ensuremath{T_{\rm eff}}}
\newcommand{\logg}{\ensuremath{\log{g}}}
\newcommand{\vsini}{\ensuremath{v \sin{i}}}
\newcommand{\feh}{[Fe/H]}
\newcommand{\logl}{\ensuremath{\log{L}}}

\newcommand{\rsun}{\ensuremath{R_\sun}}
\newcommand{\msun}{\ensuremath{M_\sun}}
\newcommand{\lsun}{\ensuremath{L_\sun}}
\newcommand{\teffsun}{\ensuremath{T_{eff,\sun}}}
\newcommand{\rhosun}{\ensuremath{\rho_\sun}}

\newcommand{\rstar}{\ensuremath{R_*}}
\newcommand{\mstar}{\ensuremath{M_*}}
\newcommand{\lstar}{\ensuremath{L_*}}
\newcommand{\loglstar}{\ensuremath{\log{L_*}}}
\newcommand{\teffstar}{\ensuremath{T_{eff,*}}}
\newcommand{\rhostar}{\ensuremath{\rho_*}}

\newcommand{\rearth}{\ensuremath{R_\earth}}
\newcommand{\mearth}{\ensuremath{M_\earth}}
\newcommand{\learth}{\ensuremath{L_\earth}}
\newcommand{\teffearth}{\ensuremath{T_{eff,\earth}}}
\newcommand{\rhoearth}{\ensuremath{\rho_\earth}}

%\newcommand{\rpl}{\ensuremath{R_{\rm P}}}
%\newcommand{\mpl}{\ensuremath{M_{\rm P}}}
%\newcommand{\lpl}{\ensuremath{L_{\rm P}}}
%\newcommand{\teffpl}{\ensuremath{T_{eff,{\rm P}}}}
%\newcommand{\rhopl}{\ensuremath{\rho_{\rm P}}}
%\newcommand{\ipl}{\ensuremath{i_{\rm P}}}

\newcommand{\rpl}{\ensuremath{R_{P}}}
\newcommand{\mpl}{\ensuremath{M_{P}}}
\newcommand{\lpl}{\ensuremath{L_{P}}}
\newcommand{\teffpl}{\ensuremath{T_{eff,{P}}}}
\newcommand{\rhopl}{\ensuremath{\rho_{P}}}
\newcommand{\ipl}{\ensuremath{i_{P}}}
\newcommand{\gpl}{\ensuremath{g_{P}}}

\newcommand{\rjup}{\ensuremath{R_{\rm J}}}
\newcommand{\mjup}{\ensuremath{M_{\rm J}}}
\newcommand{\ljup}{\ensuremath{L_{\rm J}}}
\newcommand{\teffjup}{\ensuremath{T_{eff,{\rm J}}}}
\newcommand{\rhojup}{\ensuremath{\rho_{\rm J}}}
\newcommand{\gjup}{\ensuremath{\g_{\rm J}}}

\newcommand{\msini}{\ensuremath{m \sin i}}

% ---------------------
\newcommand{\ordo}{\ensuremath{\mathcal{O}}}

% -----------------------------
\newcommand{\pack}[1]{\textsc{\lowercase{#1}}}
\newcommand{\prog}[1]{\texttt{\lowercase{#1}}}
\newcommand{\iraf}{\pack{iraf}}
\newcommand{\todcor}{\prog{todcor}}
\newcommand{\xcsao}{\prog{xcsao}}
\newcommand{\daophot}{\pack{daophot}}
\newcommand{\fihat}{\pack{fihat}}
\newcommand{\fistar}{\prog{fistar}}
\newcommand{\fiphot}{\prog{fiphot}}
\newcommand{\grmatch}{\prog{grmatch}}
\newcommand{\grtrans}{\prog{grtrans}}

% ---------------------------------------
\newcommand{\pref}[1]{p.~\pageref{#1}}
\newcommand{\figr}[1]{Fig.~\ref{fig:#1}}
\newcommand{\secr}[1]{\mbox{\S\ \ref{sec:#1}}}
\newcommand{\eqr}[1]{Eq.~\ref{eq:#1}}
\newcommand{\tabsr}[1]{Tab.~\ref{tab:#1}}
\newcommand{\tabr}[1]{\mbox{Table~\ref{tab:#1}}}
\newcommand{\figrp}[1]{Fig.~\ref{fig:#1} on \pref{fig:#1}}
\newcommand{\secrp}[1]{\S\ref{sec:#1} on \pref{sec:#1}}
\newcommand{\eqrp}[1]{Eq.~\ref{eq:#1} on \pref{eq:#1}}
\newcommand{\tabrp}[1]{Tab.~\ref{tab:#1} on \pref{tab:#1}}

% --------------------------------------
%
% Instruments
%
% FLWO 1.2 m telescope
\newcommand{\flwof}{\mbox{FLWO 1.2 m}}

% FLWO 1.5 m telescope
\newcommand{\flwos}{\mbox{FLWO 1.5 m}}

% TopHAT 0.25m telescope
\newcommand{\flwot}{\mbox{TopHAT 0.25 m}}

% MMT
\newcommand{\mmt}{\mbox{MMT 6.5 m}}

% Spitzer
\newcommand{\ssts}{{\em Spitzer}}
\newcommand{\sstL}{{\em Spitzer Space Telescope}}

% HST
\newcommand{\hst}{{\em HST}}


% --------------------------------------
% Variable types
%
\newcommand{\dscu}{\mbox{$\delta$ Scuti}}
\newcommand{\gdor}{\mbox{$\gamma$ Dor}}

\newcommand\kepler{\textit{Kepler}}
\newcommand\tess{TESS}
\newcommand{\ktwo}{\emph{K2}}
\newcommand{\plato}{PLATO}

\newcommand{\refsec}[1]{\mbox{\S\ \ref{sec:#1}}}
\newcommand{\fig}[1]{Fig.\,\ref{fig:#1}}

% --------------------------------------
% Formatting
%
\newcommand{\fancysection}[1]{
%
    \section{#1}
%
    \hrule \vspace{0.5mm}
%
}

\newcommand{\hlinesep}{\noindent\rule{\linewidth}{1pt}\\\vspace{-5mm}}

\newcommand{\subpixtools}{\pack{SuperPhot}}
\newcommand\fitsubpix{\pack{FitSubPix}}
\newcommand\fitpsf{\pack{FitPSF}}
\newcommand\subpixphot{\pack{SubPixPhot}}
\newcommand\imsubtract{\pack{ImSubtract}}

\newcommand\hasfeature{\cellcolor{green}\checkmark}
\newcommand\nofeature{\cellcolor{red!60}$\mathbf{\times}$}
\newcommand\mostlyfeature{\cellcolor{green!20!yellow}mostly}
\newcommand\somefeature{\cellcolor{orange}some}
\newcommand\partialfeature[1]{\color{red}{#1}}

\newcommand\poet{\texttt{POET}}

\newcommand\secinputdata{\refsec{input_data}}
\newcommand\sectestingtheory{\refsec{testing_theory}}
\newcommand\secmethodology{\refsec{methodology}}
\newcommand\secprelimresults{\refsec{preliminary_results}}
\newcommand\ruskin{\citep{Patel_Penev_21}}
\newcommand\ebmcmc{\citep{Windemuth_Agol_Kiefer_19}}


\usepackage[font={normal,it}]{caption}
\renewcommand{\familydefault}{\sfdefault}

\usepackage{floatrow}
\usepackage{wrapfig}
\usepackage[pdftex]{graphics,graphicx}
\usepackage{epstopdf}
\graphicspath{ {project_description/figures/} }
\usepackage{pslatex}
\usepackage[margin=2.5cm]{geometry}
\usepackage{fancyhdr}
\usepackage{sectsty}
\allsectionsfont{\normalsize}
\usepackage{natbib,amsmath,amssymb}
\usepackage{sidecap}
\usepackage{aas_macros}
\usepackage{multicol}
\usepackage{relsize}
\usepackage[bookmarks=true]{hyperref}
\usepackage{ifthen}
\usepackage[table]{xcolor}
\usepackage{rotating}
\usepackage{booktabs}

\usepackage{tabularx}
\usepackage{makecell}

\usepackage{sidecap}

\setcounter{secnumdepth}{3}
\setcounter{tocdepth}{2}
\usepackage{multirow}
\usepackage{spverbatim}
\usepackage{pdfpages}

\usepackage{enumitem}
\setlist{nosep}
\setlist[itemize]{leftmargin=*}
\setlist[enumerate]{leftmargin=*}

\usepackage[compact]{titlesec}
\titlespacing*{\section}{0pt}{0pt}{0pt}
\titlespacing*{\subsection}{0pt}{0pt}{0pt}
\titlespacing*{\subsubsection}{0pt}{0pt}{0pt}
%\titlelabel{\roman{\thetitle}.\quad}
%\titleformat{\paragraph}
%    {\normalfont\bfseries}
%    {\theparagraph\quad}
%    {0pt}
%	{\vspace{-3mm}}

\widowpenalty 0
\widowpenalties 0
\clubpenalties 1 0

\renewcommand\bibsection{}

\chead{}

\rhead{\thepage}

\lfoot{}

\cfoot{}

\rfoot{}

\newlength{\defaultparfillskip}
\setlength{\defaultparfillskip}{\parfillskip}

\newlength{\defaultparindent}
\setlength{\defaultparindent}{\parindent}

\definecolor{highlightcolor}{rgb}{1.00, 0.50, 0.50}%{0.0274, 0.7215, 0.5019}

\sethlcolor{highlightcolor}

%\renewcommand{\hl}[1]{\pdfliteral direct {2 Tr 0.5 w}#1\pdfliteral direct {0 Tr 0 w}}

\title{\vspace{-20mm}Measuring Tidal Dissipation of Low Mass Stars and Giant Planets}
\date{}


\begin{document}

\maketitle

\vspace{-15mm}

\section{Project Description}

Tides play an important role in shaping exoplanet and binary star systems. The
effects of tidal coupling are evident over the entire life cycle of binary stars
and exoplanet systems, and have been suggested to play a key role in the
formation of short period giant planets, ultra short period planets, and a wide
range of satellite systems within the Solar System. In spite of the wide spread
implications, there is limited understanding of the processes involved, and the
strength of the tidal coupling they produce. The proposed investigation aims to
provide systematic, consistent, and reliable measurements of tidal coupling over
a broad range of parameters, directly usable to model the effect of tides for
almost all known exoplanet systems.

Recent Lonestar6 and stampede2 allocations have allowed us to publish five peer
reviewed publications on this topic \citep{Mahmud_et_al_23, Patel_et_al_23,
Penev_Schussler_22, Anderson_et_al_21, Penev_et_al_18}. Four of these used, and
acknowledge using TACC resources, and one performing secondary analysis without
use of TACC resources. Results were also presented at eight national and
regional conferences, including American Astronomical Society and the American
Physical Society. Two successful PhD degrees were defended based on this work.
We have also secured grant funding from NASA to support the proposed work and
its extensions over the following three years under two separate grants. The
first (80NSSC23K1486) started Sep 2023 and the second starting May 2025.

Per the NASA grants, we request resources for two projects that we expect to
complete over the following year:

\textbf{Project 1:} We are currently in the final stages of development of the
software for a project that will study tides in a collection of 184 binary stars
that orbit each other close enough to raise significant tides on each other. The
friction these tides experience change the orbits of the binaries over time and
we plan to use these signatures to infer the amount of tidal friction present.
We will use Bayesian analysis (Markov chain Monte Carlo or MCMC) to constrain a
generic parametric model of tidal friction using observations of the
eccentricity of the orbits of these binary stars. This project was also proposed
last year, however before committing significant TACC resources we identified a
problem in the implemented calculations which forced us to reimplement a lot of
the code. We have since corrected the error and have thoroughly tested that
calculations are now correct and are in the early stages of the MCMC analysis
for the first batch of systems.

\textbf{Project 2:} We are also in the process of software development for a
project that aims to prepare an approximately 10 times larger, but otherwise
very similar catalog to the one used by project 1. The catalog will be based on
observations from space (from the TESS satellite) of the dimming of the stars as
they block each other's light. Thanks to the unprecedented precision of the
observations and the fact that TESS has observed the entire sky, this catalog
will provide a unique combination of very precisely determined physical
parameters for a very large collection of very bright stars.  We plan to use
this catalog to explore even more tidal physics (though not during the coming
year), but it will also provide invaluable input data for studies by many other
research groups as is evidenced by the much smaller catalog mentioned above.
This project will also rely on MCMC analysis, this time using models of the
eclipses and other effects detectable in these binary stars to compare to the
observed brightness as a function of time of each system by TESS.

\section{Resource Justification/Computational Plan}

We will use the \texttt{emcee}
package\footnote{\url{https://emcee.readthedocs.io/en/stable/}} for the MCMC
analysis. Both applications described above require only small amount of RAM and
rely on widely used and extensively tested/benchmarked open source packages for
parallelizing the computations.

We have developed and are already using the code for project 1 on TACC and we
expect to complete the rest of the project without any further modifications to
the code. So far we have completed about 13\% of the sampling, consuming 7000
SUs, thus we estimate that the remaining 85\% will require 45\,000 SUs.

The requirements of project 2 are far less straightforward to estimate since so
far we have a working model for eclipsing binaries, but do not yet have fully
working sampling so we are forced to estimate how many samples will be required
for the MCMC sampling to converge. A similar analysis was performed by
\citet{Windemuth_et_al_19} using data from a different satellite. They showed
$10^6$ \texttt{emcee} steps of 128 coupled MCMC chains were required for the
samples to converge to a stable posterior distribution. Since our previous
proposal we have completed building the full eclipsing binary model and are in
the process of implementing and testing the MCMC sampling. We carried out three
two hour MCMC runs on Lonestar6 with the, finding that approximately 10\,000
MCMC steps per hour are produced using 8 parallel processes per binary star. We
will run 16 stars per node each using 8 parallel processes. We found that using
up to 8 processes resulted in 7 times more steps per hour than using only 1
process. Hence, we were making efficient use of the node. We estimate 16\,000
SUs will be required to complete analysis of the approximately 2500 systems we
plan to include in the first release of the catalog.

Thus we request a combined 60\,000 SUs to support both projects described above.

\section{Progress Report}

Responding to the request from most the recent review, we provide here a brief
progress report.

\textbf{Project 1:} We have completed the software development for the project
and are now in the process of running the MCMC analysis utilizing TACC
resources. We have already finished sampling for the first 9 systems and the
rest are in progress. Our allocation for July 2023 --- June 2024 was sparsely
utilized because we had to redesign the sampling code we had implemented. As a
result we were awarded just 10\,000 SUs of the 65\,000 SUs we requested for July
2024 --- June 2025. The code is now in its final fully working state. As is
evidenced by our recent use of the existing allocation, we are confident we will
make efficient use of this allocation if approved. We expect the first
publication to be submitted within the next three months.

\textbf{Project 2:} We have completed and thoroughly tested 90\% of the software
development for the second project described above. Namely, we now have a fully
functional model for predicting what should be observed for a binary given
assumptions for its physical parameters. The remaining effort is to use that
model to calculate MCMC likelihoods and run sampling. This has allowed us to
refine the estimated SUs required to complete the project, though significant
uncertainty still remains due to the unknown number of samples required for
convergence.

\newpage

\section*{\centering{\Large References and Citations}}
\addcontentsline{toc}{section}{\protect{\numberline{\thesection}}References and
Citations}

\bibliography{bibliography}

\bibliographystyle{mnras}

\end{document}
