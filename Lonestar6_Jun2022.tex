\documentclass{article}

\usepackage[bookmarks=true]{hyperref}

\title{Measuring Tidal Dissipation of Low Mass Stars and Giant Planets}

\begin{document}

\maketitle

\section{Project Description}

Tides play an important role in shaping exoplanet and binary star systems. The
effects of tidal coupling are evident over the entire life cycle of binary stars
and exoplanet systems, and have been suggested to play a key role in the
formation of short period giant planets, ultra short period planets, and a wide
range of satellite systems within the Solar System. In spite of the wide spread
implications, there is limited understanding of the processes involved, and the
strength of the tidal coupling they produce. The proposed investigation aims to
provide systematic, consistent, and reliable measurements of tidal coupling over
a broad range of parameters, directly usable to model the effect of tides for
almost all known exoplanet systems.

Thanks to past allocations on stampede 2 we have been able to complete all
calculations for three publications submitted to refereed journals, as well as
present results at four conferences, and work is on-going on three additional
publications, which we expect to complete within the next 6 months.

\section{Resource Justification/Computational Plan}

We plan to perform Bayesian analysis to constrain a generic parametrized tidal
dissipation model against observations of the eccentricity and spin
distributions of binary star systems. In particular we will use Markov Chain
Monte Carlo, using the \texttt{emcee}
package\footnote{\url{https://emcee.readthedocs.io/en/stable/}}. We have the
code for two projects we plan to complete over the next year fully developed and
have been using it over the past 2 years on stampede2 and an HPC cluster at UT
Dallas called Ganymede.


Our applications require only small amount of RAM and rely on widely used and
extensively tested/benchmarked open source packages for parallelizing
computations. We request a start-up allocation on the Lonestar6 system in order
to learn to use that system equally efficiently by the time stampede2 is
decommissioned, and also in order to get reliable measurements on how quickly we
accumulate MCMC steps so we can estimate reliably the computational requirements
for an eventual research allocation request when the proposal window opens.


A naive scaling based on threads and clock rate compared to stampede 2, and the
currently completed computations suggest we will ultimately need more than
60\,000 SUs on Lonestar6 in addition to the 15\,000 SUs we were recently awarded
on Stampede2 to complete the computations of the three projects we are currently
working on. These were computed as follows:

Using 19\,000 SUs on stampede2, together with additional (unmetered)
computations on the UT Dallas Ganymede cluster, we recently completed the
analysis of the eccentricity of 38 binary stars. We plan to apply the exact same
analysis to 93 additional systems. Scaling by 96 threads of stampede2 node at
2.1 GHz vs 128 threads at 2.45 GHz for Lonestar6, we estimate this project will
require 30\,000 SUs on Lonestar6. 

A different analysis method was applied to measured spins of 44 binary stars,
consuming 10\,000 SUs on Stampede2 + unknown but small contribution from
Ganymede. We are currently ready (all input data is collected) to apply exactly
the same analysis to 99 additional systems. Again scaling by threads and clock
rate, we expect this will consume $\sim14\,000$ SUs on Lonestar6.

Finally, we plan to carry out eccentricity analysis of 75 exoplanet systems in
order to measure the tidal friction in gas giant planets. Calculating
evolutions of exoplanet systems takes similar time to calculating binary star
evolutions, hence we can scale the 19\,000 Stampede2 SUs to analyze 38 systems
mentioned above to estimate we need $\sim24\,000$ Lonestar6 SUs for this
analysis.

Finally, we subtract the already awarded $15\,000$ SUs on stampede2, which
approximately scale to $9\,600$ equivalent SUs on Lonestar6 to arrive at our
final estimate of additional $\sim60\,000$ Sus on Lonestar6 required to complete
all 3 projects

\end{document}
